\subsubsection*{Tytuł pracy} 
\Title

\subsubsection*{Streszczenie} 
Niniejsza praca inżynierska zajmuje się lokalizacją, opartą o system GPS. Jej celem jest śledzenie obiektów. Aby go zrealizować, stworzona została aplikacja, która ma możliwość wyświetlania tras przebytych przez konkretne obiekty, takie jak lokalizator, kierowca lub pojazd. Praca opisuje proces budowania programu oraz zawiera niezbędne informacje, które powinna zawierać dokumentacja tego typu projektu. W pierwszej kolejności temat został przeanalizowany. Następnie wybrano odpowiednie narzędzia, stos technologiczny oraz sformułowano wymagania funkcjonalne i niefunkcjonalne. Kolejnym krokiem było sporządzenie specyfikacji, zarówno zewnętrznej, opisującej wygląd aplikacji, sposób obsługi oraz wymagania systemowe, jak i wewnętrznej, składającej się z opisu architektury i struktur danych, popartych licznymi schematami i diagramami. W tej części znajduje się również wytłumaczenie działania niektórych fragmentów kodu. Przeprowadzono testy aplikacji, aby wyeliminować powstałe w trakcie tworzenia kodu błędy. Na końcu sformułowano wnioski i porównano efekty do wyznaczonych wcześniej celów. 

\subsubsection*{Słowa kluczowe} 
Lokalizator, pojazd, kierowca, aplikacja, program

\subsubsection*{Thesis title} 
\begin{otherlanguage}{british}
\TitleAlt
\end{otherlanguage}

\subsubsection*{Abstract} 
\begin{otherlanguage}{british}
This thesis deals with location, based on a GPS system. Its aim is to track objects. In order to make it happen, an application that has the ability to display routes traveled by specific objects, such as tracker, driver or vehicle was created. Thesis describes the process behind building the program and it contains necessary information that should be included in the documentation of this type of project. First, the topic was analysed. Next, the right tools were chosen, as well as technology stack and functional and non-functional requirements were written down. The next step was to prepare specifications, both external, which describes user interface, manual and system requirements, and internal, which consists of a description of the architecture and data structures, surrounded with numerous diagrams. There are also some explanations for fragments of code in this section. Tests were carried out to eliminate errors in application, which could have appeared during coding. At the end, conclusions were made and the effects were compared to the previously set goals.
\end{otherlanguage}
\subsubsection*{Key words}  
\begin{otherlanguage}{british}
Tracker, vehicle, driver, application, program
\end{otherlanguage}

