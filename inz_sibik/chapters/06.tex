\chapter{Weryfikacja i walidacja}
\label{ch:06}

\section{Sposób testowania}
\paragraph{}
Aby przetestować działanie serwera oraz prawidłowość w konfiguracji lokalizatora wykorzystano maszynę wirtualną z systemem Linux Ubuntu, która posiadała publiczny adres IP. Zakupiono lokalizator Mking MK07A oraz kartę SIM z pakietem internetu oraz wysyłką wiadomości SMS. Następnie urządzenie zostało skonfigurowany poprzez SMS-y tak, aby wysyłało dane na adres maszyny wirtualnej z użyciem portu 8080. Jednak serwer w języku Java był dostępny na lokalnym komputerze stacjonarnym z systemem Windows. Aby przekierować pakiety na serwer, skorzystano z programu MobaXTerm, który umożliwia tunelowanie portów TCP. Kolejnym korkiem było zatrzymywanie serwera w odpowiednich miejscach w programie IntelliJ Idea, dzięki czemu następnie eliminowano błędy wynikające z niewłaściwej interpretacji pakietów oraz poprawiano kod serwera.

\paragraph{}
W celu sprawdzenia aplikacji pod kątem błędów przechodzono przez jej zakładki, a w nich przez formularze. Testowano ich pola oraz czy zostają one przesłane, podczas gdy brakuje wymaganych pól.

\section{Wykryte i usunięte błędy}
\paragraph{}
Podczs testów aplikacji wykryto, że formularze pozwalają wpisać w pola dat zakończenia daty wcześniejsze niż te będące już w polach dat rozpoczęcia. Rozwiązano ten błąd poprzez wprowadzenie walidacji na pola z datami, aby data do była musiala być późniejszą datą niż data od.

