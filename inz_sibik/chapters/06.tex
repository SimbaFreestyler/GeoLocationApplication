\chapter{Weryfikacja i walidacja}
\label{ch:06}

\paragraph{}
Testowanie jest nieodzowną częścią podczas prac nad projektem. Pozwala wyeliminować błędy, powstałe na skutek implementacji skomplikowanych obliczeń czy też niedopatrzenia ze strony autora. Zrezygnowanie z tej czynności wiąże się z dużo większym prawdopodobieństwem istnienia problemów związanych z aplikacją, a to znacząco zwiększa ryzyko stworzenia niekompletnego programu.

\section{Sposób testowania}
\paragraph{}
Aby przetestować działanie serwera oraz prawidłowość w konfiguracji lokalizatora w pierwszej kolejności skonfigurowano router oraz komputer lokalny, celem wykorzystania przekierowania portów. Napotkano problem, związany z dostawcą internetu, co uniemożliwiało odbieranie danych z lokalizatora przez router. Następnie zainstalowano narzędzie Ngrok, które umożliwia przekierowanie portów oraz udostępnia publiczną domenę. Znaleziono problem w tym sposobie testowania, gdyż publiczna domena zmieniała adres IP w bardzo krótkim czasie. Po skonfigurowaniu lokalizatora na nowy adres, czyli wysłaniu odpowiedniej komendy SMS z nowym adresem, adres IP otrzymywany przez Ngrok ponownie się zmieniał, co uniemożliwiło otrzymanie pakietów TCP od urządzenia GPS. Ostatecznie wykorzystano maszynę wirtualną z systemem Linux Ubuntu, która posiadała publiczny adres IP. Zakupiono lokalizator Mking MK07A oraz kartę SIM z pakietem internetu oraz wysyłką wiadomości SMS. Następnie urządzenie zostało skonfigurowany poprzez SMS-y tak, aby wysyłało dane na adres maszyny wirtualnej z użyciem portu 8080. Jednak serwer w języku Java był dostępny na lokalnym komputerze stacjonarnym z systemem Windows. Aby przekierować pakiety na serwer, skorzystano z programu MobaXTerm, który umożliwia tunelowanie portów TCP. Kolejnym korkiem było zatrzymywanie serwera w odpowiednich miejscach w programie IntelliJ Idea, dzięki czemu następnie eliminowano błędy wynikające z niewłaściwej interpretacji pakietów oraz poprawiano kod serwera.

\paragraph{}
W celu sprawdzenia aplikacji pod kątem błędów skorzystano z testowania modelem V. Jest to model sekwencyjny, który składa się z testów takich jak testy integracyjne, systemowe czy akceptacyjne. Podczas testów integracyjnych, przechodzono przez zakładki aplikacji, sprawdzając poprawność działania. Ważnym aspektem programu były formularze, zatem im również poświecono uwagę - testowano ich pola, oraz zastosowane w nich walidacje, w tym czy dane zostają przesłane, podczas gdy formularz jest niepoprawnie wypełniony i zawiera błędy. Sprawdzone zostało również działanie aplikacji na ekranach różnej wielkości, aby upewnić się, czy aplikacja jest uniwersalna i dostosowana do urządzeń mobilnych.

\section{Wykryte i usunięte błędy}
\paragraph{}
Podczas testów aplikacji wykryto, że formularze pozwalają wpisać w pola dat zakończenia daty wcześniejsze niż te będące już w polach dat rozpoczęcia. Rozwiązano ten błąd poprzez wprowadzenie walidacji na pola z datami, aby data do była musiała być późniejszą datą niż data od.

