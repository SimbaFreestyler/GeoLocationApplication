\chapter{Podsumowanie i wnioski}

\paragraph{}
Projekt pozwolił autorowi na zgłębienie wiedzy na temat lokalizatorów GPS oraz ich działania, jak również dotyczącej komunikacji za pomocą protokołu TCP. Poznano wiele narzędzi do zarządzania połączeniami sieciowymi, takich jak Ngrok oraz MobaXTerm. Jednoosobowe budowanie aplikacji przeglądarkowej pozwoliło na rozwój w kierunku programisty typu fullstack - oznaczającego specjalizowanie się w pełnym stosie technologicznym, który obejmuje: backend, frontend i bazę danych.

\section{Uzyskane wyniki w świetle postawionych celów i zdefiniowanych wymagań}
\paragraph{}
Aplikacja została napisana zgodnie z wymaganiami, które zostały przedstawione. Odpowiednio skonfigurowany lokalizator - co ułatwia instrukcja dostępna w programie - poprawnie współpracuje z serwerem, w wyniku czego można korzystać z aplikacji w sposób zgodny z początkowymi celami. Prosty interfejs pozwala użytkownikowi na sprawne i przyjemne obsługiwanie programu. Widok został dostosowany do różnej wielkości ekranów, co wskazuje na sukces podczas wprowadzania skalowalności. Wszystkie operacje, które miały być dostępne do wykonania przez użytkownika, i które miały na celu osiągnięcie użytecznej aplikacji, zostały wprowadzone.

\section{Kierunki dalszego rozwoju systemu}
\paragraph{}
Dalszym etapem rozwijania aplikacji byłoby przede wszystkim zwiększenie dostępnych typów lokalizatorów, wspieranych przez program. W tym celu należy rozbudować kod serwera, odbierającego dane geograficzne od urządzeń. Przykładem innego lokalizatora jest Coban TK108, który również umożliwia wysyłanie pakietów TCP na niestandardowy adres IP oraz port. Takie działania zwiększyłyby grupę potencjalnych użytkowników aplikacji, gdyż nie byliby oni ograniczeni do zakupu jednego modelu lokalizatora.
