\chapter{Specyfikacja wewnętrzna}
\label{ch:05}

\subsection{Przedstawienie idei}
\paragraph{}
Widok użytkownika napisany został w języku TypeScript, bazuje na komponentach biblioteki React. Podłoże aplikacji jest oparte na języku Java. Znanym wzorcem projektowym jest MVC (ang. Model View Controller), co przetłumaczone na język polski brzmi: model, widok, kontroler. Model to cześć odpowiadająca za przechowywanie danych. Widok to część dostępna dla użytkownika aplikacji, natomiast kontroler wykonuje operacje i jest odpowiedzialny za logikę programu. Do stworzenia programu wykorzystano architekturę warstwową, która bazuje na MVC oraz skorzystano z wzorca projektowego o nazwie repozytorium. Repozytorium oznacza oddzielenie logiki biznesowej od bazy danych, natomiast architektura wartwowa zakłada istnienie takich warstw jak: encje - będące reprezentacją danych, kontrolery - obsługujące żądania HTTP, serwisy - odpowiedzialne za logikę biznesową i będące wywoływanymi przez kontrolery, repozytoria - mające dostęp do bazy danych, wykonujące operacje zapisu lub pobrania danych. Mapowanie encji na tabele w bazie danych, jak i odwrotnie, jest dostępne dzięki zastosowaniu adnotacji Hibernate.

\subsection{Opis struktur danych}
\paragraph{}
Dane w kodzie języka Java są reprezentowane przez encje. Są to klasy posiadające pola - zmienne odzwierciedlające kolumny w tabelach w bazie danych. Ich obiekty są odwzorowaniem bazodanowych rekordów. Przykład kodu można zobaczyć ......... Każda zmienna posiada adnotację (poprzedzoną znakiem "@"), jest to składania Hibernate. Dzięki temu jest możliwe mapowanie między obiektem encji, a rekordem w tabeli. Encje posiadają również funckje nazywane getterami - do pozyskiwania danych z konkretnych pól oraz setterami - do wpisywania wartości do pól.

\paragraph{}
Aby wysyłać i odbierać dane od tzw. frontendu, czyli części aplikacji dotyczącej interfejsu użytkownika, niezbędne są klasy reprezentujące żądania (ang. request) - odbierane są przez kontrolery oraz odpowiedzi (ang. response) - wysyłane dane do części z językiem TypeScript. Zazwyczaj mają pola analogiczne do encji. Posiadają również funkcje pobierające wartości z konkretynch zmiennych oraz zapisujące wartości do nich, tak jak jest to w przypadku encji. Odpowiedzi są również wyposażone w funkcje konwertujące obiekt encji na obiekt odpowiedzi. Przykład tej klasy znajduje się .......


Jeśli „Specyfikacja wewnętrzna”:
\begin{itemize}
\item przedstawienie idei
\item architektura systemu
\item opis struktur danych (i organizacji baz danych)
\item komponenty, moduły, biblioteki, przegląd ważniejszych klas (jeśli występują)
\item przegląd ważniejszych algorytmów (jeśli występują)
\item szczegóły implementacji wybranych fragmentów, zastosowane wzorce projektowe
\item diagramy UML
\end{itemize}

% % % % % % % % % % % % % % % % % % % % % % % % % % % % % % % % % % % 
% Pakiet minted wymaga importu: \usepackage{minted}                 %
% i specjalnego kompilowania:                                       %
% pdflatex -shell-escape main                                       %
% % % % % % % % % % % % % % % % % % % % % % % % % % % % % % % % % % % 


Krótka wstawka kodu w linii tekstu jest możliwa, np.  \lstinline|int a;| (biblioteka \texttt{listings})% lub  \mintinline{C++}|int a;| (biblioteka \texttt{minted})
. 
Dłuższe fragmenty lepiej jest umieszczać jako rysunek, np. kod na rys \ref{fig:pseudokod:listings}% i rys. \ref{fig:pseudokod:minted}
, a naprawdę długie fragmenty – w załączniku.


\begin{figure}
\centering
\begin{lstlisting}
class test : public basic
{
    public:
      test (int a);
      friend std::ostream operator<<(std::ostream & s, 
                                     const test & t);
    protected:
      int _a;  
      
};
\end{lstlisting}
\caption{Pseudokod w \texttt{listings}.}
\label{fig:pseudokod:listings}
\end{figure}

%\begin{figure}
%\centering
%\begin{minted}[linenos,frame=lines]{c++}
%class test : public basic
%{
%    public:
%      test (int a);
%      friend std::ostream operator<<(std::ostream & s, 
%                                     const test & t);
%    protected:
%      int _a;  
%      
%};
%\end{minted}
%\caption{Pseudokod w \texttt{minted}.}
%\label{fig:pseudokod:minted}
%\end{figure}


