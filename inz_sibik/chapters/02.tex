\chapter{Analiza tematu}

\paragraph{}
Lokalizatorów samochodowych na rynku jest wiele. W zależności od ceny, możemy otrzymać dodatkowe funkcjonalności, mniej lub bardziej znaczące, są to między innymi czujnik wstrząsu, monitorowanie prędkości czy podsłuch. Producenci zazwyczaj posiadają własne strony internetowe, do których klient otrzymuje bezpłatny dostęp po zakupie produktu. Pozwalają one zwykle powiązać lokalizator z kontem użytkownika i śledzić na bieżąco jego lokalizację.

\paragraph{}
W celu stworzenia aplikacji do monitorowania lokalizatorów dla jak największej liczby ludzi, korzystających z tego typu urządzeń, warto przeanalizować kilka kwestii. Niezbędną funkcją lokalizatora jest możliwość jego konfiguracji, aby wysyłał dane na konkretny adres IP oraz port. Umożliwi to dostęp aplikacji do lokalizacji urządzenia. Z tego powodu odrzucono produkty o najniższej cenie na rynku, gdyż nie oferują one wymaganej do działania programu funkcji. Kolejnym znaczącym aspektem wyboru lokalizatora jest jego cena w kontekście klienta. Należy wykluczyć najdroższe opcje, aby nie ograniczyć ilości potencjalnych użytkowników aplikacji. Uwzględniając powyższe wymagania, wybrano model Mking MK07A. Dodatkowym atutem jest jego akumulator, którego pojemność wynosi 10000 mAh, dzięki czemu nie wymaga częstego ładowania.

\paragraph{}
Podczas przeprowadzania analizy tematu, należało również wybrać stos technologiczny, w którym aplikacja będzie tworzona. Ze względu na liczne biblioteki usprawniające proces rozwijania oprogramowania, część funkcjonalną programu postanowiono napisać w języku Java. W celu konfiguracji aplikacji przeglądarkowej została wybrana platforma programistyczna, jaką jest Spring Boot - jedna z bardziej popularnych opcji przy tego typu programach pisanych w języku Java. Drugą częścią, która zostanie stworzona, jest część interfejsu. W tym przypadku wybrano bibliotekę React, będącą powszechnym narzędziem, oferującym wiele przydatnych funkcji. Biblioteka oparta będzie na języku TypeScript - jest to rozszerzenie języka JavaScript, które wymaga określania typów zmiennych, dzięki czemu można uniknąć dużej liczby błędów w przyszłości, które pojawiają się w JavaScript przy braku typowania. Należy się również zastanowić nad przechowywaniem danych. Służą do tego bazy danych. W niniejszej aplikacji w celu tworzenia i obsługi takiej bazy zostanie użyty język PostgreSQL. Jest to uwarunkowane danymi, które pojawią się w systemie, a mianowicie dane geograficzne - długość i szerokość geograficzna. PostgreSQL posiada rozszerzenie - PostGIS - umożliwiające zapis długości i szerokości geograficznej w jednej kolumnie, jako punkt na Ziemii.



