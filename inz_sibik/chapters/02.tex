\chapter{Analiza tematu}

\paragraph{}
Lokalizatorów samochodowych na rynku jest wiele. W zależności od ceny, możemy otrzymać dodatkowe funkcjonalności, mniej lub bardziej znaczące, są to między innymi czujnik wstrząsu, monitorowanie prędkości czy podsłuch. Producenci zazwyczaj posiadają własne strony internetowe, do których klient otrzymuje bezpłatny dostęp po zakupie produktu. Pozwalają one zwykle powiązać lokalizator z kontem użytkownika i śledzić na bieżąco jego lokalizację.

\paragraph{}
W celu stworzenia aplikacji do monitorowania lokalizatorów dla jak największej liczby ludzi, korzystających z tego typu urządzeń, warto przeanalizować kilka kwestii. Niezbędną funkcją lokalizatora jest możliwość jego konfiguracji, aby wysyłał dane na konkretny adres IP oraz port. Umożliwi to dostęp aplikacji do lokalizacji urządzenia. Z tego powodu odrzucono produkty o najniższej cenie na rynku, gdyż nie oferują one wymaganej do działania programu funkcji. Kolejnym znaczącym aspektem wyboru lokalizatora jest jego cena w kontekście klienta. Należy wykluczyć najdroższe opcje, aby nie ograniczyć ilości potencjalnych użytkowników aplikacji. Uwzględniając powyższe wymagania, wybrano model Tk108. Dodatkowym atutem jest jego akumulator, którego pojemność wynosi 10000 mAh, dzięki czemu nie wymaga częstego ładowania.

\begin{itemize}
\item sformułowanie problemu
\item osadzenie tematu w kontekście aktualnego stanu wiedzy (\english{state of the art}) o poruszanym problemie
\item  studia literaturowe \cite{bib:artykul,bib:ksiazka,bib:konferencja,bib:internet} -  opis znanych rozwiązań (także opisanych naukowo, jeżeli problem jest poruszany w publikacjach naukowych), algorytmów, 
\end{itemize}


Wzory  
\begin{align}
y = \frac{\partial x}{\partial t}
\end{align}
jak i pojedyncze symbole $x$ i $y$  składa się w trybie matematycznym.


%%%%%%%%%%%%%%%%%%%%%%%%



