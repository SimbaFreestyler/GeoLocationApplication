\chapter{Wstęp}
\label{ch:wstep}

Lokalizatory GPS w wielu dziedzinach są wsparciem dla ludzi. Szczególnie ciekawym aspektem jest temat lokalizatorów w pojazdach. Podczas kupna takiego urządzenia, klient otrzymuje zazwyczaj dostęp do storny internetowej, na której jest w stanie sprawdzać położenie swojego samochodu, w którym został umieszczony lokalizator. Takie rozwiązania są dosyć proste, niewystarczające dla wielu użytkowników, wygląd interfejsu również pozostawia wiele do życzenia. Wraz z rozwojem komputerów oraz smartfonów, zwiększają się możliwości stworzenia aplikacji do zarządzania lokalizatorami w pojazdach, która byłaby atrakcyjna dla 
Tematem niniejszej pracy inżynierskiej jest aplikacja przeglądarkowa pozwalająca na monitorowanie położenia geoprzestrzennego obiektów przy użyciu lokalizatorów. 

\begin{itemize}
\item wprowadzenie w problem/zagadnienie
\item osadzenie problemu w dziedzinie
\item cel pracy
\item zakres pracy
\item zwięzła charakterystyka rozdziałów
\item jednoznaczne określenie wkładu autora, w przypadku prac wieloosobowych – tabela z autorstwem poszczególnych elementów pracy
\end{itemize}

