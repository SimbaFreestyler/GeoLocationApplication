\chapter{Wstęp}
\label{ch:wstep}

\paragraph{}
Początki technologii do określania pozycji sięgają lat 60. XX wieku. Powstał wtedy system NAVSAT (ang. Navigation Satellite System) - będący pierwszym satelitarnym systemem nawigacyjnym. Został on oprcaowany przez Stany Zjednoczone oraz wykorzystywany był przez tamtejszą marynarkę wojenną. W latach 70. XX wieku postanowiono wprowadzić międzynarodowy standard, dzięki czemu powstał system GPS (ang. Global Positioning System), który jest używany po dziś dzień.

\paragraph{}
Długi czas obecności tego systemu na rynku zaowocował jego rozwojem, jak również dostępnością dla przeciętnych użytkowników. W efekcie tego, GPS jest wsparciem dla ludzi w wielu dziedzinach. W obecnych czasach ponad połowa światowej populacji posiada smartfony, które to mają wbudowane systemy GPS. Pozwala to przede wszystkim na sprawną nawigację do celu czy  dokładne ustalenie pozycji danej osoby. Samochody również są w posiadaniu znacznej części populacji, a ponadto stanowią dosyć znaczną część budżetu domowego. Z tego względu ludzie zaopatrują się w lokalizatory samochodowe. Podczas kupna takiego urządzenia, klient otrzymuje zazwyczaj dostęp do strony internetowej, na której jest w stanie sprawdzać położenie swojego samochodu, w którym został umieszczony lokalizator. Takie rozwiązania są dosyć proste, niewystarczające dla wielu użytkowników, wygląd interfejsu również pozostawia wiele do życzenia. Wraz z rozwojem komputerów oraz smartfonów, zwiększają się możliwości do stworzenia aplikacji do zarządzania lokalizatorami w pojazdach, która oferowałaby większą ilość funkcjonalności, oraz która byłaby atrakcyjniejsza niż podstawowe odpowiedniki.

\paragraph{}
Celem niniejszej pracy inżynierskiej jest stworzenie aplikacji przeglądarkowej pozwalającej na zarządzanie lokalizatorami, pojazdami i kierowcami oraz możliwość wyświetlania ich tras na mapie. Praca obejmuje proces i sposób tworzenia oprogramowania, specyfikacje: zewnętrzną i wewnętrzną,  testowanie, jak również efekty i wnioski.

%\begin{itemize}
%\item wprowadzenie w problem/zagadnienie
%\item osadzenie problemu w dziedzinie
%\item cel pracy
%\item zakres pracy
%\item zwięzła charakterystyka rozdziałów
%\item jednoznaczne określenie wkładu autora, w przypadku prac wieloosobowych – tabela z autorstwem poszczególnych elementów pracy
%\end{itemize}

