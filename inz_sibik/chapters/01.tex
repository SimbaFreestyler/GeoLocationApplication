\chapter{Wstęp}
\label{ch:wstep}
\paragraph{}
Ten rozdział ma na celu przybliżenie czytelnikowi zagadnienia systemów lokalizacji, będących ściśle związanych z tematem pracy, jak również przedstawienie zakresu, celu pracy oraz krótki opis kolejnych rozdziałów, które niniejsza praca inżynierska zawiera.

\section{Wprowadzenie w problem}

\paragraph{}
Początki technologii do określania pozycji sięgają lat 60. XX wieku. Powstał wtedy system NAVSAT (ang. Navigation Satellite System) - będący pierwszym satelitarnym systemem nawigacyjnym. Został on opracowany przez Stany Zjednoczone oraz wykorzystywany był przez tamtejszą marynarkę wojenną. W latach 70. XX wieku postanowiono wprowadzić międzynarodowy standard, dzięki czemu powstał system GPS (ang. Global Positioning System), który jest używany po dziś dzień. Więcej na temat jego historii znajduje się w książce Jeffa Hurna "GPS: A Guide to the Next Utility"\ \cite{bib:gps}.

\paragraph{}
Przez lata system ten był rozwijany, co zaowocowało jego dostępnością dla przeciętnych użytkowników. W efekcie tego, GPS jest wsparciem dla ludzi w wielu dziedzinach. W obecnych czasach ponad połowa światowej populacji posiada smartfony, które to mają wbudowane systemy GPS. Pozwala to przede wszystkim na sprawną nawigację do celu czy  dokładne ustalenie pozycji danej osoby. Samochody również są w posiadaniu znacznej części populacji, a ponadto stanowią dosyć znaczną część budżetu domowego. Z tego względu ludzie zaopatrują się w lokalizatory samochodowe. Podczas kupna takiego urządzenia, klient otrzymuje zazwyczaj dostęp do strony internetowej, na której jest w stanie sprawdzać położenie swojego samochodu, w którym został umieszczony lokalizator. Takie rozwiązania są dosyć proste, niewystarczające dla wielu użytkowników, wygląd interfejsu również pozostawia wiele do życzenia. Wraz z rozwojem komputerów oraz smartfonów, zwiększają się możliwości do stworzenia aplikacji do zarządzania lokalizatorami w pojazdach, która oferowałaby większą ilość funkcjonalności, oraz która byłaby atrakcyjniejsza niż podstawowe odpowiedniki, będące obecnie na rynku.

\section{Cel pracy}
\paragraph{}
Celem niniejszej pracy inżynierskiej jest śledzenie lokalizacji obiektów. Umożliwi to aplikacja, pozwalająca na zarządzanie lokalizatorami, pojazdami i kierowcami, która usprawniłaby obługę tego typu urządzeń w obrębie rodziny lub firmy posiadającej flotę samochodów i ułatwiłaby przeglądanie tras przebytych przez poszczególne osoby, pojazdy czy lokalizatory.

\section{Zakres pracy}
\paragraph{}
 Praca obejmuje proces i sposób tworzenia oprogramowania, specyfikacje: zewnętrzną i wewnętrzną, testowanie, jak również efekty i wnioski.

\section{Charakterystyka rodziałów}
\paragraph{}
Następnym rozdziałem jest rozdział drugi, będący zatytuowanym "Analiza tematu". Jego treść ukazuje proces rozezniania autora w temacie lokalizatorów, jak również porównianie planowanego projektu do istniejących rozwiązań. Znajduje się tam też proces wyboru stosu technologicznego. Kolejny rozdział - trzeci - nosi nazwę "Wymagania i narzędzia". Pierwszą jego część stanowią wymagania, które dzielą się na funkcjonalne oraz niefunkcjonalne. Obydwa ich rodzaje są wymienione i opsiane. Drugą częścią rozdziału są narzędzia. Autor przedstawia w niej niezbędne urządzenia, programyoraz wybrany stos technologiczny, niezbędny do zbudowania projektu. "Specyfikacja zewnętrzna"\ jest rozdziałem czwartym. Znajduje się tam przede wszystkim opis widoku systemu połączony z instrukcją obsługi. W zrozumieniu tekstu pomagają umieszczone na rysunkach zrzuty ekranu, przedstawiające wygląd konkretnych fragmentów programu. W rozdziale czwartym sformułowane są także wymagania sprzętowe i programowe, stanowiące o urządzeniach, na których aplikacja będzie w stanie zostać uruchomiona, a także przedstawione są rodzaje użytkowników systemu oraz funkcje, do których mają dostęp. Piąty rozdział ma tytuł "Specyfikacja wewnętrzna". W tej części pracy opisane zostaly struktury i architektura danych, które zostały wykorzystane w projekcie, a zatem również przedstawiona została baza danych, poparta schematem. Rozdział ten posiada też opis fragmentów kodu źródłowego projektu. Następnym rozdziałem - szóstym - jest "Weryfikacja i walidacja". Dotyczy on przeprowadzania testów i sprawdzeń pod kątem błędów, które mogły się pojawić podczas tworzenia programu. Ostatni, siódmy rozdział zatytuowany jest "Podsumowanie i wnioski". W nim zostały przedstawione zostały efekty uzyskane w projekcie w porównaniu do wstępnych założeń, jak również przemyślenia autora, które wytworzyły się podczas prac nad aplikacją oraz niniejszą pracą inżynierską.   